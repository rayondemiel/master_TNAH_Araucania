\chapter{Introduction}
\chaptermark{Introduction}

\epigraph{\enquote{La masse des choses dites dans une culture, conservées, valorisées, réutilisées, répétées et transformées. Bref, toute cette masse verbale qui a été fabriquée par les hommes, investie dans leurs techniques et leurs institutions, et qui est tissée avec leur existence et leur histoire}}{\textit{Michel Foucault}\protect\footnotemark}
\footnotetext{\cite[Michel Foucault, « La naissance d’un monde », in][texte n°68, p.~814-815]{foucaultDitsEcrits195419881994a}}

Une querelle persiste depuis le milieu du XX\textsuperscript{e} sur la notion \enquote{archive} entre l'archéologie foucaldienne et les institutions patrimoniales. La seconde la définit comme le produit ontologique d'une action de stockage, de préservation et de classification de documents divers. Dans une autre optique, Michel Foucault lui apporte une nouvelle dimension, celle d'une reconstitution d'un discours dans sa matérialité historique. C’est \enquote{le système général de la formation et de la transformation des énoncés \footcite[Michel Foucault, L’archéologie du savoir, Paris, Gallimard, 1969, p. 177-179. \textit{via}]{ogilvieParadoxesArchive2017a}}. \par
Le souci culturel et intellectuel de conserver cette \enquote{masse des choses} et l'éclosion des outils numériques au sein des institutions patrimoniales et des sciences humaines ont permis une réconciliation de ces deux dimensions contemporaines de l'archive. La mutation de cette "masse" en \enquote{\textit{data}} pouvant être à la fois être conservée, disséquée et contextualisée, voire même enrichie, permet l'acquisition de nouvelles compétences aux institutions patrimoniales et scientifiques. L'automatisation des processus de traitements documentaires (dans le sens archivistique ou éditorial) a fait l'objet d'un véritable investissement, devenant une priorité d'action de développement au sein de nombreux centres ces dix à vingt dernières années \footcite{MichelFoucaultNumerique2021a}. Les nouvelles méthodes informatiques donnent la capacité de renouveler les méthodes d'exploration des documents et donner un nouveau sens à l'information. Elles permettent un accroissement de l'offre de valorisation considérable\footnote{Les récents projets des archives nationales répondent à ce besoin de transcender la fonction archivistique traditionnelle à travers le numérique à l'image du projet d'édition numérique des testaments de poilus ou du projet NER4Archives. \cite{clavaudVersEditionLigne2019a}; \cite{clavaudNER4ArchivesNamedEntity2022}}. Le numérique est devenu un outil indispensable à la fois au scientifique humaniste comme à l'archiviste face à l'accroissement exponentiel de bases de données dédiées à l'exploration de corpus documentaire.\newpar

Toutefois, cette réalité numérique reste concentrée aux plus grandes puissances économiques. Cette transformation numérique fait encore face à de nombreuses disparités géographiques; de nombreux pays et institutions sont cantonnés à la marge de cette transformation numérique de par les besoins matériels, financiers et techniques que cela exige \footcite{wissikTeachingDigitalHumanities2020a}. Si l'archive reste le symbole de l'affirmation de l'État et de ses besoins bureaucratiques, la pérennisation des instruments culturels reste bien souvent un domaine sacrifié. \par

Le Chili, pays bien souvent considéré comme le plus développé d'Amérique du Sud au vu de ses infrastructures politiques, économiques et sanitaires, reste le symbole de ce contraste du déploiement du numérique au sein des institutions culturelles\footcite{undpNextFrontierHuman2020a}. Un contraste à la fois continental où les projets numériques se concentrent aux pays de la pointe du continent, mais où les initiatives numériques restent encore marginales, mais aussi avec les pays occidentaux. Un groupe de chercheurs pointent justement un système concentrique autour des projets anglo-saxons, puis européens puis le reste du monde\footcite{russellGeographicalLinguisticDiversity}.\newpar

Néanmoins, les humanités numériques et l'accroissement des projets numériques suscitent de plus en plus d'intérêt au sein des institutions patrimoniales et universitaires chiliennes. Dans ce cadre, le centre \gls{acab} a déployé un certain nombre de projets autour de la valorisation numérique de leurs ressources archivistiques, plus particulièrement sur la numérisation, mais aussi dernièrement une volonté de s'inscrire dans le processus de \glspl{lod}, (données ouvertes et liées) notamment autour des archives du célèbre poète et député communiste chilien Pablo Neruda\footnote{Note interne de service.}.\par

Le centre \gls{acab} a pris naissance officiellement en 1994, bien que son héritage institutionnel remonte à la première partie du XIX\textsuperscript{e} siècle comme bibliothèque de l'\textit{Instituto Nacional} (Institut Nationale)\footnote{\textit{Archivo Central Andres Bello}, url: \url{http://archivobello.uchile.cl/acerca-del-archivo/historia}, consulté le 17/02/2022.}. Il appartient à l'Universidad de Chile, situé à Santiago, qui est la plus grande université publique du Chili. La structure est sous la responsabilité du Bureau du vice-recteur pour la vulgarisation et la communication et elle est divisée en trois aires de compétences :

\begin{itemize}
\item \textit{Información Bibliográfica y Archivística} (Information bibliographie et archivage) : Le service est en charge de l'inventorisation et la classification des collections tout en assurant l'accès aux documents pour le public.

\item \textit{Conservación y Patrimonio} (Conservation et patrimoine): l'équipe est responsable de la préservation (biochimique) et de la restauration des différents documents abîmés. Il assure aussi l'accès au musée et la numérisation des collections.

\item \textit{Investigación Patrimonial} (Recherche patrimoniale): Il s'agit de la partie scientifique et éditoriale du centre afin d'exploiter les documents possédés. Des programmes éducatifs sont mis en place par l'équipe afin de valoriser les fonds documentaires.
\end{itemize}
Comme nous venons de le voir, le centre culturel est avant tout une structure pluridisciplinaire alliant des compétences éditoriales, archivistiques, conservations et scientifiques. C'est plus d'une vingtaine de personnes qui travaillent ainsi dans les locaux de la maison centrale de l'Universidad de Chile. Au total, \gls{acab} gère plus de 18 collections documentaires, dont trois sont classés comme \enquote{Monument National} par le ministère de l'Éducation. \newpar
Suite à plusieurs échanges et du fait de l'intérêt soucieux pour les nouvelles technologies et leurs apports aux compétences patrimoniales, un projet est né autour des archives de l'\enquote{\textit{Ocupación de la Araucanía}\footnote{\enquote{Occupation de l'Araucanie}. Suite à de nombreuses contestations sociales, mais aussi scientifiques, le terme a substitué la dénomination de \enquote{Pacificación de la Araucanía} (Pacification de l'Araucanie).}} (1850-1883). Cette volonté est double puisqu'il s'agît de faire une première prospective de l'intérêt des outils numériques et notamment de l'apprentissage machine comme appui aux politiques patrimoniales. Dans un second temps, il s'agit de mettre en place un processus d'édition nativement numérique de ces archives.\par
Dans ce cadre, un stage de 4 mois entre avril et juillet 2022 a été mis en place au sein de \gls{acab} et l'Universidad de Chile avec la collaboration et le soutien de l'\gls{enc} ainsi que de la Région Île-de-France afin de développer un certain nombre d'outils permettant l'automatisation de l'édition numérique des archives autour de l'Occupation de l'Araucanie.



%http://humanidadesdigitales.net/