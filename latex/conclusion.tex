\chapter*{Conclusion}
\chaptermark{Conclusion}
\addcontentsline{toc}{chapter}{Conclusion}

Ce mémoire et ce stage ont été l'occasion de mettre en évidence ou de confirmer certaines appréciations sur le développement d'une chaîne de traitement automatique pour la production d'édition numérique native. De la production d'un modèle \gls{htr} à l'analyse, l'annotation et l'enrichissement des éditions restent un processus complexe qui doit s'inscrire dans une dynamique globale. Le projet sur les archives autour de l' \enquote{Occupation de l'Araucania} est particulièrement révélateur de ces attentes.

Si la scientificité des données repose l'apport quantitatif, mais aussi critique que le chercheur y apporte. Il en reste que l'automatisation du processus de traitement représente des avantages considérables pour les institutions patrimoniales et scientifiques, sans pour autant prétendre à une exhaustivité. Le travail de l'ingénierie se tâche de concevoir et d'ajuster du mieux possible les différents programmes afin de procéder au traitement le plus complet possible, mais qui se confrontent bien souvent à de nombreuses problématiques techniques et intellectuelles.\newpar

En premier lieu, la mise en place d'un tel projet nécessite de mettre en œuvre un travail conséquent de préparation de données sur lequel va se centrer l'ensemble du programme. Il faut donc pouvoir saisir l'essence des documents qui constituent notre réservoir et notre objet d'études. L'ensemble de cette \textit{pipeline} se nourrit de ses données afin de pouvoir alimenter, poursuivre et améliorer le rendu de celles-ci. L'apprentissage automatique est ainsi devenu un axe majeur de notre réflexion et de notre pratique. Son introduction doit être considérée non pas comme finalité, et ce malgré la plus-value esthétique que ces techniques représentent, mais comme un moyen avec ces faiblesses, ces biais et ses réussites.

Ces méthodes d’apprentissages ont permises de numériser un grand nombre de nouvelles données sous l'encodage \gls{tei} et ce à partir de données océrisées. Les technologies \gls{htr} ont été des outils efficaces pour la numérisation native des documents historiques et ainsi servir leur pérennité. Pour cela, la mise en place d'un prototype d'une transformation vers le format pivot a été nécessaire.

Toutefois, il en est ressorti de nombreuses difficultés quant à leurs traitements postérieurs, ce qui induit le besoin de nouvelles réflexions sur les données présentant des défis variations orthographiques historiques. La faible homogénéité des données produites a amenée à utiliser et mettre en place un modèle de reconnaissance des entités nommées afin de procéder à l'indexation des entités majeures et laisser l'opportunité d'explorer plus en profondeur les documents ou de l'enrichir. De même, l'automatisation de la segmentation reste une étape difficile à automatiser. Cette chaîne de traitement ne peut se passer de l'intervention directe de l'humain afin d'en contrôler la qualité et la cohérence.

En observant le filigrane de ce projet, nous avons pu de même observer l'enjeu que représente l'ouverture des données, de la cohérence de leurs partages et la mise en place de plateformes communes. Ces initiatives permettent une véritable mutualisation des efforts permettant de soutenir des projets de moindre envergure qui peuvent s'appuyer sur des programmes plus vastes. En outre, ces données ouvertes interrogent sur notre rapport à la construction et l'appropriation de la science \enquote{en adoptant une posture humble et ouverte devant nos sources, sans appropriation indue et sans dissimuler la possibilité de l’erreur — mieux, en fournissant à la communauté non seulement les résultats de nos recherches, mais aussi la manière dont ils ont été produits et peuvent être reproduits\footcite{campsOuVaPhilologie2018}}. Ces données doivent pour s'émanciper de l'autorité productrice afin de dévoiler de nouvelles perceptions au travers de nouveaux regards éthiques, méthodologiques ou socio-politiques.\newpar

Le projet de ce stage a conclu au développement d'un mille-feuille dont chaque étape ne prend sens que collectivement si qui nécessite un travail en amont. Le numérique ne peut s'éloigner ou s'émanciper des méthodes humanistes, devant au contraire s'accommoder afin de faire face aux enjeux du \textit{big data} ou des nouvelles méthodes numériques. Pour Dominique Boullier, les sciences sociales ne peuvent pas être le fruit d'une numérisation de leurs méthodes, de leurs données et de leur objet d’étude, mais en apporter une valeur ajoutée\footcite{boullierSociologieNumerique2016}. Cette volonté d'enrichir s'appuie sur cette volonté d'utiliser ces nouvelles méthodes afin d'inciter ces nouvelles perceptions épistémologiques, mais aussi des nouveaux enjeux autour de la valorisation numérique. Víctor Gayol et Jairo Antonio Melo Flórez appellent, justement, les instituions a développer ces projets, notamment autour de l'histoire digital afin de sortir des expériences individuelles allant au détriment d'une véritable cohérence\footcite{gayolPresentePerspectivasHumanidades2017}.

Ces humanités numériques  restent en plein essor et entourés d'espoirs sur la compréhension d'une histoire complexe ou la mémoire est encore douloureuse. Du Chili au Mexique, ce champs des humanités s'organise et se construit progressivement afin de faciliter la circulation des savoir-faire et des données à l'image du réseau RED\footnote{Red de Humanidades Digitales, url: \url{http://humanidadesdigitales.net/}, consulté le 30/08/2022}. La multiplication des efforts conjoints et de mise en parallèle avec des projets vont permettre révéler et encourager les archives autour de l'\enquote{Occupation de l'Araucanie} afin d'encourager l'émergence d'une nouvelle historiographie sur ces évènements\footcite{canalestapiaHistoriografiaMapucheBalances2015}.

